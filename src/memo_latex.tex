\documentclass[a4paper, 12pt, french]{article}
\usepackage[utf8]{inputenc} %encodage
\usepackage[T1]{fontenc} %taille de fonte
\usepackage{babel} %Facilitation du francais
\usepackage{amsmath}

\title{Memo LaTeX}
\author{Elanis - https://github.com/Elanis/studies-cheatsheet}

\begin{document}
	\maketitle

	Utilisation de \emph{l'emphase}.

	\section{Listes}

	\begin{itemize}
		\item Premier
		\item Second

		Second paragraphe du second
	\end{itemize}

	\begin{enumerate}
		\item Premier
		\item Second
	\end{enumerate}

	\begin{description}
		\item[Premier] choix
		\item[Second] choix
	\end{description}

	\section{Notes}
	\label{sec:notes}

	Pour faire des notes, nous pouvons utiliser plusieurs commandes. L’une d’elle permet 
	d’avoir des notes de marges, les deux autres des notes de bas de page.

	Ceci est un document avec une note de bas de page ici\footnote{voici la note}.

	Ceci est un document avec une note dans la marge \marginpar{voici la note}.

	\section{Les références}
	Dans ce document, nous avons parlé des notes dans la section \ref{sec:notes} qui est à 
	la page \pageref{sec:notes}.

	\section{Maths}

	\subsection{inline}

		test de maths \no 1(le plus utilisé) $f(x) = 2x$

		test de maths \no 2 \(f(x) = 2x\)

		test de maths \no 3 \begin{math} f(x) = 2x \end{math}

	\subsection{block}

		Premier (le plus utilisé)
		\[
			f(x) = 2x
		\]

		Second
		\begin{displaymath}
			f(x) = 2x
		\end{displaymath}

	\subsection{notations}
		$2^2$
		$2_2$

		$2^{10}$
		$2_{50}$

		$O^8_{450}$

		$\frac{12}{34}$

		$\sqrt{2}$

		$cos(x)$ %probleme
		$\cos(x)$ %solution

		$\sin(x)$
		$\sinh(x)$
		etc.

		\[
			\sum_{k = 0}^{n} k = \frac{n(n+1)}{2}
		\]

		\[
			\lim_{x \to \infty} f(x) = \ell
		\]

		%necessite amsmath
		\[
			\binom{n}{k} \text{ est le nombre de parties $k$ dans un ensemble à $n$ elements}
		\]

		\[ %probleme
			( \frac{1}{2} )
		\]

		\[ %solution
			\left( \frac{1}{2} \right)
		\]


\end{document}