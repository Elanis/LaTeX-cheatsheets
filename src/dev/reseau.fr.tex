\documentclass[a4paper, 12pt, french]{article}
\usepackage[utf8]{inputenc}
\usepackage[T1]{fontenc}
\usepackage{babel}

%Images
\usepackage{graphicx} 
\graphicspath{{src/dev/}}

%Code
\usepackage{listings}
\usepackage{color}

\definecolor{dkgreen}{rgb}{0,0.6,0}
\definecolor{gray}{rgb}{0.5,0.5,0.5}
\definecolor{mauve}{rgb}{0.58,0,0.82}

\lstset{frame=tb,
  language=C,
  aboveskip=5mm,
  belowskip=5mm,
  framesep=3mm,
  showstringspaces=false,
  columns=flexible,
  basicstyle={\small\ttfamily},
  numbers=none,
  numberstyle=\tiny\color{gray},
  keywordstyle=\color{blue},
  commentstyle=\color{dkgreen},
  stringstyle=\color{mauve},
  breaklines=true,
  breakatwhitespace=true,
  tabsize=3
}

%Commandes perso
\newcommand{\hr}{\noindent\rule{13.7cm}{0.4pt}}

%Metas
\title{Réseau}
\author{Elanis - https://github.com/Elanis/LaTeX-cheatsheets}
\date{}

\begin{document}
	\maketitle

	\includegraphics[width=13.8cm]{reseau_modele_osi}

	\section{Introduction}

	« Un réseau est un ensemble d'équipement reliés entre eux et qui échangent des informations »

	\subsection{Échanges sur de longues distances}
	\begin{itemize}
		\item Satellite
		\item Hertzien (TV)
		\item Cable aerien
		\item Antenne Parabolique
		\item Paire torsadée (ADSL, Telephone, etc)
		\item WiMax, Telephonique (2G, 3G, 4G), etc
		\item ...
	\end{itemize}

	\subsection{Échanges sur de courtes/moyennes distances}
	\begin{itemize}
		\item CPL (Courant porteur de ligne): Passage du signal par le réseau éléctrique
		\item NFC (Sans contact): Telephones, Cartes bancaires, Badges, etc
		\item Radio: Bluetooth, WiFi, etc
		\item Femtocell: mini-relai telephonique
		\item Ethernet (de 10 Mbps à 10 Gbps)
		\item ...
	\end{itemize}
	
	\subsection{Internet Mobile}  
	\begin{description}
		\item[2G] GSM/GPRS/EDGE - 384Kbps $\downarrow$ - 188.4Kbps $\uparrow$ - 900-1800 Mhz
		\item[3G] UMTS/3G+ (HSPA+) - 14.4Mbps $\downarrow$ - 5.7Mbps $\uparrow$ - 1900-2100 Mhz
		\item[4G] 4G (LTE) - 100Mbps $\downarrow$ - 50Mbps $\uparrow$ - 800 - 1800 - 2600 Mhz
	\end{description}

	\subsection{Internet Fixe}  
	\begin{description}
		\item[ADSL] 13.7Mbps $\downarrow$
		\item[ADSL 2] 25Mbps $\downarrow$
		\item[VDSL 2] 120Mbps $\downarrow$
		\item[Fibre optique] 200Mbps $\downarrow$
		\item[Internet par satellite] 20Mbps $\downarrow$ - \emph{Note:} Utilisée principalement dans les zones non reliables par les solutions précédentes.
	\end{description}

	\subsection{Le Cloud}

	Le but du cloud est de "dématerialiser" les systèmes informatiques en les envoyant sur le Web. Il en résulte des problèmes sur le réseau avec tout ce nouveau traffic qui auparavant était local aux réseaux d'entreprise le plus souvent. Il exite néanmoins des solutions:

	\subsubsection{Le Edge Computing}

	L'edge computing est une méthode d'optimisation employée dans le cloud computing qui consiste à traiter les données à la périphérie du réseau, près de la source des données. Il est ainsi possible de minimiser les besoins en bande passante entre les capteurs et les centres de traitement des données en entreprenant les analyses au plus près des sources de données.

	\subsubsection{Le Fog Computing}

	Le fog computing consiste à exploiter des applications et des infrastructures de traitement et de stockage de proximité, servant d'intermédiaire entre des objets connectés et une architecture cloud classique.

	\subsection{Niveaux de service}

	\includegraphics[width=13.8cm]{reseau_niveaux_de_service}

	\section{Couche Physique}

	\emph{Bande passante:} Taille de la bande de frequence utilisée pour transmettre des données

	\emph{Attenuation: } Diminution du signal sur la distance

	\subsection{Supports de transmissions}

	\emph{Particularités diverses: } Bande passante, attenuation, sensibilité, coût, facillité d'installation, etc

	\begin{description}
		\item[Paire torsadée:] dans un même câble par 2, 4 ou 8 fils; une paire est un lien de communication. Chaque paire est enroulée pour limiter les interferences. \emph{Debit Max: 100 Gbps}. Simple, economique, reutilisable, mais sensible aux pertubations électromagnetiques et grande atténuation.
		\item[Coaxial:] il consiste en deux conducteurs ayant le même axe. \emph{Debit Max: 2 Gbps} Tres bonne qualité, haut debit, bonne manipulation mais très cher.
		\item[Fibre optique:] Cylindre de fibre de sillicium extremement fin. \emph{Debit Max: plusieurs terabits par seconde} Bande passante immense, debits très importants, insensibles aux interferences et corrosions chimiques, attenuation faible, leger mais fragile, unidirectionnel, transmission point à point, cout élevé des interfaces.
		\item[Ondes radio:] Wifi (56Mbps), Bluetooth (2Mbps), Infrarouge, Hertzien, etc \emph{Portée jusqu'a quelques centaines de mètres}
		\item[Micro ondes:] Transmission terrestre de 50 à 1000km et satellitaire a 36000km (geostationnaire) ou 800 km d'altitude
	\end{description}


	\subsection{Transmission}

	\subsubsection{Caracteristiques d'un signal}

	Chacun se decompose en une somme infinie de sinusoides (decompostion de Fourrier)

	\emph{Note:} Tout ce qui est hors de la bande passante est naturellement filtré.  

	Rapport signal sur bruit(dB) $ = 10 * log_{10} \frac{S}{B}$ où S est la puissance du signal et B celle du bruit.

	\subsubsection{Capacité d'un canal de transmission}

	\emph{Theoreme de Shannon:} Soient W la bande passante d'un canal de transmission et S/B le rapport signal sur bruit. La capacité en bit/s du canal est :

		$$C = W * log_2 \left( 1 + \frac{S}{B} \right)$$

	\subsection{Numerisation, Quantification, Echantillonnage}

	\emph{Analogique:} Signal continu

	\emph{Numerique:} Signal echantilloné (sur le temps), et quantifié (sur les valeurs) afin d'être stockées et transmises par un equipement electronique.

	\subsubsection{Echantillonnage}

	\emph{Theoreme de Shannon:} Prendre comme max 2 fois la frequence max relevée

	\subsubsection{Quantification}

	Utilisation d'echelles logarithmiques

	\subsubsection{Formules}


	\begin{description}
		\item[Debit binaire ($D_b$)] Nombre max d'elements binaires transmis par seconde
			$$D_r = 1/T_b \ bit/s$$
		\item[Rapidité de modulation ($R_s$)] Vitesse à la queles les symboles se succèdent
			$$ R_s = 1/T_s  \ bauds$$
		\item[Valence (V)] Cardinal de l'alphabet des symboles
			$$ D = R_s * r = R_s * log_2 * V $$
		\item[Debit binaire max] Donne le debit binaire maximum en fonction de la bande passande du canal
			$$ D_{max} = 2 * W * log_2 * V $$
			où W est la largeur de la bande passante et V la valence du signal
	\end{description}

	\subsection{Transmission en bande de base}

	On transmet en bande de base, quand on transmet directement l'information codée sous forme d'un signal carré.

	\begin{description}
		\item[Tout ou rien] 0 est 0 Volt, 1 est +x Volts
		\item[Bipolaire] 0 est 0 Volts, 1 est alternativement +x ou -x Volts. \emph{Permet de mieux distinguer les suites de 1}
		\item[NRZ (No return to zero)] 0 est -x Volts, 1 est +x Volts. \emph{Permet de differencier le silence des 0}
		\item[NRZI (NRZ Inverted space)] 0 est un changement de niveau (entre +x et -x Volts), 1 correspond à une absence de changement.
		\item[RZ (Return to zero)] 0 est 0 Volts, 1 est un front montant (impulsion au milieu du temps bit)
		\item[Biphasé (ou Manchester)] 0 est un front montant, 1 est un front descendant. \emph{Il est plus facile a distinguer grace a un changement de signe à chaque bit}
		\item[Code manchester différentiel] 0 est un front inversé au cycle précédent, 1 est le même front que le cycle précédent.
		\item[Code Miller] 0 est une absence de changement sur l'intervalle, 1 est un front montant/descendant sur l'intervalle
 	\end{description}

	\subsection{Transmission modulée}

	Le principal problème de la transmission en bande de base est que le signal se degrade sur la distance, il est donc utilisé uniquement sur des réseaux locaux. On utilise donc pour de plus grandes distances des signaux sinuisoidaux générés et recupérés par des modems (modulateur - demodulateur).

	Afin d'y introduire de l'information, les modulations permettent de transformer:
	\begin{itemize}
		\item L'amplitude
		\item La fréquence
		\item La phase
	\end{itemize}

	\subsubsection{Modulation d'amplitude}

	Il s'agit d'associer un symbole à chaque amplitude.

	\emph{Avantages:} Moins couteux et plus précis qu'un signal carré, la modulation d'amplitude est simple.
	
	\emph{Désavantages:} Sensibles aux perturbations électromagnétiques (orage, ligne éléctrique, ...)

	\subsubsection{Modulation de fréquence}

	Il s'agit d'associer un symbole à chaque fréquence.

	\emph{Avantages:} Moins couteux et plus précis qu'un signal carré, résistant aux pertubations (d'amplitude).
	
	\emph{Désavantages:} Système de démodulation moins simple à concevoir

	\subsubsection{Modulation de phase}

	Il s'agit d'associer un symbole à des changements de phases.

	\emph{Avantages:} Les dispositifs de (dé)modulation de phase permettent de coder facilement deux états, résistant aux pertubations (d'amplitude).
	
	\emph{Désavantages:} Système de démodulation pas simple à concevoir

	\subsubsection{Combinaison de modulations}

	Les transmissions modulées peuvent combiner plusieurs formes de modulations simultanées. On représente ces modulations par un diagramme spatial:

	\includegraphics[width=13.8cm]{reseau_diagramme_spatial}

	\section{Couche Liaison}

	Le but de la couche liaison est de transporter les trames d'un nœud à un autre de manère fiable.\\

	\emph{DCE: } Detection et correction d'erreur

	La couche liaison offre 2 fonctions:
	\begin{itemize}
		\item Detection et correction d'erreurs
		\begin{itemize}
			\item Bits de parités
			\item Code Cyclique
		\end{itemize}
		\item Protocoles d'accès au canal
		\begin{itemize}
			\item Protocoles à partitionnement du canal
			\item Protocoles aléatoires
			\item Protocoles à partage de ressources
		\end{itemize}
	\end{itemize}

	Les services de la couche liaison sont:
	\begin{description}
		\item[Encapsulation] incorporer les données de la couche supérieur (réseau) dans une trame
		\item[Accès au médium] protocole d'accès au canal de communication
		\item[Détection / Correction d'erreurs] mécanismes de détection / correction d'erreurs de transmission
		\item[Transmission fiable] service d'acquittement et de retransmission pour certains liens peu fiables (ex. sans fil)
		\item[Contrôle de flux] éviter les surcharges si trop de trames arrivent trop vite sur un nœud
	\end{description}

	Cette couche est souvent impantée sur la carte réseau et est autonome.

	\subsection{Detection et correction d'erreur}

	\subsubsection{Bits de parités}

	Ajouter aux données un bit DCE afin de fixer la parité de la trame (Fixer le nombre de 1 dans la trame: pair ou impair).

	\emph{Limitations:} Le nombre d'erreurs detectable est au maximum de 1, en effet, s'il y'a 2 erreurs, la parité s'annule.

	\subsubsection{Bits de parités bidimensionneles}

	Le principe est le même que le précédent, sauf que les bits de parités sont insérés en ligne et en colonne.

	\emph{Limitations:} Permet de detecter et corriger une erreur, de detecter 2 erreurs sans pouvoir forcement les corriger.

	\emph{Note:} Ce même principe est utilisé dans les lecteurs CD audio

	\subsubsection{Codes cycliques CRC}

	\emph{Principe:}
	\begin{itemize}
		\item On désire envoyer une trame \emph{D} de \emph{d} bits, à laquelle on ajout un DCE \emph{R} de \emph{r} bits
		\item Les deux cotés se mettend d'accord sur une suite de \emph{r+1} buts qu'on appellera \emph{G} pour générateur
		\item L'émetteur envoie le message qui contient \emph{(D, R) (d + r bits)} de telle sorte que le message soit divisible par \emph{G}.
	\end{itemize}

	\emph{Verification:} Le récepteur recoit le message, le divise par \emph{G}, et si il reste 0 il n'y a pas d'erreur.\\

	\includegraphics[width=8cm]{reseau_crc_principe}

	\emph{Remarques:}
	\begin{itemize}
		\item En ajoutant \emph{R} à \emph{D}, on crée un message divisible par \emph{G}, c'est pour cela que \emph{R} est le reste de la division.
		\item Les CRC sont normalisés selon la taille de leur générateur. Exemple: CRC-32 a un générateur de 32 bits
		\item Chaque CRC peut detecter des erreurs de largeur \emph{r} bits ou moins.
	\end{itemize}

	\subsection{Protocoles d'accès au canal}

	Il existe deux types de liens: point à point (deux nœuds se partagent un canal), ou diffusion (canal partagé par plusieurs nœuds). Sur un canal partagé, les nœuds qui transmettent des trames en même temps, ce qui peut engender des \emph{collisions}. Ces collisions brouillent les trames les rendant illisibles.

	\subsubsection{Protocoles à partionnement du canal}

	Il s'agit de diviser la bande passante et/ou le temps de parole equitablement pour chaque nœud.

	\includegraphics[width=10cm]{reseau_partitionnement_canal}

	\subsubsection{Protocoles à accès aléatoires}

	Tout les nœuds profitent de tout le débit et transmettent quand ils le desirent. S'il detectent une collision, il retransmettent après un temps aléatoire. Le processus est réitéré jusqu'a ce que la trame soit transmise.

	\subsubsection{Protocoles ALOHA à allocation temporelle}

	Le temps est divisé en intervalles egaux pour tous, où transmettre une trame. Les nœuds sont synchronisés temporellement.\\

	\emph{Deroulement:}
	\begin{itemize}
		\item Attendre le début du prochain intervalle pour transmettre une trame
		\item Si pas de collisions, on passe à la trame suivante (puis revenir au début)
		\item Si il y a eu collision
		\begin{itemize}
			\item avec probabilité p on retransmet la trame (puis revenir au début)
			\item avec probabilité (1-p) on laisse passer l'intervalle (puis revenir au début)
		\end{itemize}
	\end{itemize}

	\includegraphics[width=13.8cm]{reseau_aloha}

	\subsubsection{Protocoles ALOHA pur}

	Il n'y a plus de synchronisation du temps, le nœud transmet la trame dès qu'il la recoit des couches supérieures.\\

	\emph{Deroulement:}
	\begin{itemize}
		\item S'il n'y a pas de collision, on passe à la trame suivante
		\item Si il y a eu collision
		\begin{itemize}
			\item avec probabilité p on retransmet la trame
			\item avec probabilité (1-p) attendre un moment (le temps nécessaire pour transmettre une trame)
			\item revenir à 1
		\end{itemize}
	\end{itemize}

	\subsubsection{Protocole CSMA/CD - Carrier Sense Multiple Access with Collision Detection}

	\emph{Deroulement:}
	\begin{itemize}
		\item On écoute le canal
		\begin{itemize}
			\item si il n'y a pas de trame qui transite, on envoie la trame
			\item si il y a une trame, on attend un temps aléatoire avant d'écouter à nouveau
		\end{itemize}
	\end{itemize}\mbox{}\\

	Si une fois qu'on a commencé à émettre, on détecte une collision, on arrête de transmettre et on attend un temps aléatoire.

	\subsubsection{Protocoles à partage de ressources}

	\emph{Deux propriétés désirables pour un protocole d'accès:}
	\begin{itemize}
		\item si un seul nœud actif, lui donner tout le débit (\emph{R} bits/s)
		\item si \emph{M} nœuds actifs, leur assurer un débit moyen de \emph{R/M} bits/s
	\end{itemize}\mbox{}\\

	ALOHA ET CSMA possèdent la première propriété mais pas la secondent. De plus, les latences de temps introduisent des intervalles ou rien ne transite, et donc une perte de débit.

	\subsubsection{Protocole à sondages}

	Un nœud maitre sont les autres nœuds à tour de rôle afin de leur indiquer qu'il peuvent transmettre.
	\emph{Deroulement:}
	\begin{itemize}
		\item le maître sonde le \emph{i} ème nœud et lui indique qu'il peut transmettre jusqu'à \emph{p} trames
		\item Si le nœud a des trames à transmettre, il le fait, pendant que le nœud maître attende
		\item Une fois terminé, le maître passe au \emph{i+1} ème nœud
	\end{itemize}\mbox{}\\

	\emph{Avantages:} Pas de collisions, pas de temps d'inactivité dans le reseau
	\emph{Désavantages:} Un delai de sondage (s'il n'y a qu'un seul nœud actif, le nœud maitre les sondes quand même tous) et si le maitre tombe en panne, il n'y a plus de transmissions.

	\subsubsection{Protocole à passage de jetons}

	Un nœud attend d'avoir le jeton pour transmettre. Le jeton est un signal (chaîne de bits) particulier qui passe entre les nœuds.

	\emph{Deroulement:}
	\begin{itemize}
		\item Un nœud prend le jeton du canal
		\item s'il a des trames à transmettre
		\begin{itemize}
			\item il peut en envoyer un nombre maximum
			\item il remet le jeton sur le canal et le transmet au nœud voisin
			\item s'il lui reste des trames, il attend le prochain passage du jeton
		\end{itemize}
		\item sinon il envoie directement le jeton à son voisin
	\end{itemize}\mbox{}\\

	\emph{Avantages:} Très efficace

	\emph{Désavantages:} Si un nœud tombe en panne, le jeton ne transite plus. Si un nœud "oublie" de remettre le jeton sur le lien, la transmissions s'arrête. Un nouveau jeton doit être régénéré.

	\subsection{Classifications de réseaux}

	\begin{description}
		\item[WAN] \emph{wide area network} échelle d'un pays, ou planétaire - composé de plusieurs opérateurs
		\item[MAN] \emph{metropolitan area network} échelle d'une ville ou d'une région - généralement un seul opérateur
		\item[LAN] \emph{local area network} échelle d'une institution (entreprise, administration, école, ...) - généralement connecté à Internet à travers un routeur
		\item[PAN] \emph{personal area network} échelle de quelques mètres
	\end{description}

	\subsection{Topologies de réseaux}

	\includegraphics[width=13.8cm]{reseau_topologies}

	\subsection{MAC (ou adresse physique)}

	Adresse unique pour tout équipement, généralement 6 octets: les 3 premiers sont un code constructeur et les 3 derniers le numéro de série.

	\emph{Exemples:} 00:22:15:50:55:4C, 00:30:48:81:66:3E

	\emph{Remarque:} FF:FF:FF:FF:FF:FF est l'adresse de diffusion sur tous les nœuds

	\subsection{Réseaux Ethernet}

	C'est une technologie concue dans les 70 qui est maintenant majoritaire dans la conception de reseaux. Elle consiste en une spécification de la couche liaison et de la couche physique.\\

	\emph{Propriétés:}
	\begin{itemize}
		\item permet d'avoir de gros débit (10Mbits/s jusqu'à 10Gbits/s)
		\item peu onéreux et facile à mettre en place
		\item utilisation de CSMA/CD (simple et décentralisé)
		\item implemente des topologies en étoile où en bus
		\item peut être utilisé sur des câbles coaxiaux, des paires torsadées, de la fibre optique, etc
	\end{itemize}

	\emph{Remarques:}
	\begin{itemize}
		\item il n'y a pas de "hadshake" (une fois envoyé, c'est oublié)
		\item si le destinataire rejete la trame, l'envoyeur n'est pas notifié
	\end{itemize}

	\subsubsection{Trame Ethernet}

	\includegraphics[width=13.8cm]{reseau_trame_ethernet}

	\emph{Remarques sur la trame:}
	\begin{itemize}
		\item le préambule sert à réveiller le receveur et à synchroniser les horloges (recaler sur le même débit).
		\item les deux derniers bits du préambule « 11 » indiquent le début des données.
		\item le type est le protocole réseau utilisé par la couche supérieure
		\item s'il y'a moins de 46 octets de données, on ajoute du bourrage pour les atteindre
		\item le CRC utilisé est un CRC-32
	\end{itemize}

	\subsubsection{Spécification physique d'Ethernet}

	La transmission est faite en \emph{bande de base}, en utilisant le \emph{Codage Manchester}. L'accès au médium se fait en CSMA/CD, plus il y'aura de collisions, plus on attendra.

	\begin{description}
		\item[10 BASE 2]
		\begin{itemize}
			\item 10 Mbit/s
			\item câbles coaxial fin
			\item topologie en bus, longueur maximale 185 m
		\end{itemize}
		\item[10 ou 100 BASE T]
		\begin{itemize}
			\item 10 ou 100 Mbit/s
			\item câbles à UTP
			\item Topologie en étoile, longueur d'un brin 100 mètres
		\end{itemize}
		\item[Ethernet Gbit ou 1000 BASE …]
		\begin{itemize}
			\item 1 ou 10 Gbit/s
			\item Initialement fibre optique, maintenant des câbles UTP
			\item Topologie en étoile, longueur d'un brin moins de 100 mètres
		\end{itemize}
		\item[100 Gbit/s] En développement
	\end{description}

	\subsection{Materiels réseaux}

	Dans un reseau il existe de nombreux éléments, qui sont soit terminaux (les nœuds) ou de connexions (repeater, hub, bridge, switch, etc).

	\subsubsection{Les hubs (concentrateurs)}

	C'est une "multiprise" simple: il dispose de plusieurs interfaces de connexions, tous les signaux recus sont diffusés aux autres interfaces.

	\subsubsection{Les switchs (commutateurs)}

	Le switch ressemble beaucoup au hub si ce n'est qu'il envoie uniquement les données sur la bonne interface s'il sait ou se trouve la destination (via une table de commutation), ou par defaut - comme le hub - sur toutes les autres interfaces. La table de commutation est construite de manière itérative en fonction des découvertes du réseau faites par le commutateur.\\

	\emph{Principe de la commutation-diffusion} Une trame arrive sur l'interface \emph{I} avec comme adresse de destination \emph{X}, le commutateur consulte sa table :
	\begin{itemize}
		\item Si \emph{X} est sur le même segment que \emph{I} -> Rejeter la trame (elle sera forcement reçu par \emph{X} )
		\item Si le segment du nœud \emph{X} est connu -> Commutation vers l'interface du segment de \emph{X}
		\item Sinon -> Diffusion sur toutes les interfaces sauf \emph{I} 
	\end{itemize}

	Il faut noter que la table de commutation est vidée après un certain temps afin d'être reactif aux changements de topologie.

	\subsection{Liaisons sans fil}

	\subsubsection{Trame sans fil}

	\includegraphics[width=13.8cm]{reseau_trame_sans_fils}

	\subsubsection{Fonctionnement}

	Service avec connexion: RTS (Request to send), CTS (Clear to send), ACK (acknowledge). Il n'y a pas de collisions car les trames contiennent la durée de transmissions, de plus le CTS est recu par tout les nœuds qui sont donc au courant de la transmission.

	\subsection{Autres protocoles de la couche liaison}

	Tous ces réseaux définissent une structure de trame ainsi que des protocoles d'accès au canal et toutes les fonctionnalités de la couche liaison.\\

	\begin{description}
		\item[PPP (point to point protocol)] Protocoles pour liaisons point-à-point 
		\item[ATM (asynchronous transfer mode)] Utilisé dans plusieurs réseaux fédérateurs d'Internet
		\item[Relais de trames (amélioration de X.25)] Utilisé par des entreprises comme alternative à Internet, Commutation de paquets sur circuits virtuels
		\item{Etc.}
	\end{description}

	\section{Couche Réseau}

	Le but de la couche réseau est d'assurer la transmission entre un nœud source et une nœud destination à travers les différents routeurs. Elle offre deux fonctions:

	\begin{itemize}
		\item Routage : le choix de l'itinéraire à emprunter. Fait par des algorithmes de routage
		\item Réexpédition : la transmission d'un paquet entrant vers une liaison sortante
	\end{itemize}

	\subsection{Généralités}

	\begin{description}
		\item[Sur le nœud source] La couche réseau récupère des messages de la couche transport, pour chaque message, elle construit un (ou plusieurs) paquet(s), la couche réseau envoie chaque paquet à la couche liaison.
		\item[Sur chaque nœud intermédiaire (routeur)] La couche réseau récupère les paquets de la couche liaison, pour chacun d’entre eux, elle construit un nouveau paquet, la couche réseau envoie chaque paquet à la couche liaison.
		\item[Sur le nœud destination] La couche réseau récupère des paquets de la couche liaison, elle extrait les données de chaque paquet et les envoie à la couche transport.
	\end{description}

	Au niveau de la couche reseau, il existe deux modes de communication: le modèle avec connexion (plutôt utilisé par les opérateurs de réseaux), le modèle sans connexion (plutôt le choix de la communauté internet).

	\subsubsection{Modèle orienté connexion (circuit victuel)}

	On peut faire l'analogie avec les circuits physiques téléphoniques, cette connexion doit être établie avant tout envoie, et la route est calculée à chaque connexion. Enfin, chaque paquet comprends la référence du circuit virtuel (ce qui explique comment la facturation finale est étabile). Ce modèle est plus efficace car ne nécessite pas de recalcul de la route à chaque paquet.

	\subsubsection{Modèle sans connexion (datagramme)}

	Les paquets sont transportés de facon indépendante et sont appelés \emph{datagrammes} (par analogie au télégramme), ils comprennent l'adresse de destination et nécessitent un système d'addressage. Ce modèle est plus souple car permet à chaque paquet d'emprunter le meilleur chemin, ce qui est interessant en cas de congestion ou de panne.

	\subsection{Routage}

	Le routage est utilisé uniquement en mode sans connexion. Il consiste à calculer la meilleur route pour chaque paquet. Les équipements utilisés pour effectuer le routage sont les \emph{routeurs}. Le meilleur chemin est celui qui passe par les liens les moins onéreux en terme de longueur, débit, côut, etc.

	Le but d'un algothime de routage est de trouver la meilleur route pour un paquet. Cet algorithme doit être exact, simple, robuste (adapté aux pannes), stable (convergence vers un état equilibré), juste (vis à vis de tout les usagers), optimal. Il existe deux classes d'algorithmes de routage:

	\begin{description}
		\item[Algorithmes de routage global] Calculer en se basant sur une information globale (le graphe entier)
		\item[Algorithmes de routage décentralisés] Calculer en se basant sur des informations locales: pas de connaissance globale du graphe, chaque nœud connaît les coût vers ses voisins auxquels il est directement connecté, les nœuds échangent donc les informations avec leurs voisins.
		\item[Dans les deux cas, l'algorithme peut être:]
		\begin{itemize}
			\item Statique: les parcours changent très peu et quasiment uniquement après intervention humaine
			\item Dynamiques: les parcours s'adaptent automatiquement aux changements de topologie
		\end{itemize}
	\end{description}

	\subsubsection{Algorithme par états de lien (Dijkstra)}

	Le graphe complet est connu de tous, tous les nœuds ont la même information. Le principe est de calculer le meilleur chemin de tous les noeuds à tout les autres avec l'algorithme de Dijkstra.

	\paragraph{Algorithme de Dijkstra}\mbox{}\\

	\emph{Notations:}
	\begin{description}
		\item[c(i, j)] du lien entre i et j
		\item[D(v)] coût courant du chemin de la source à v
		\item[P(v)] nœud précédant v dans le chemin de la source à v
		\item[N] ensemble des nœuds dont on connaît le coût minimal
		\item[Adj(i, j )] vrai si i et j sont adjacents
	\end{description}

	\begin{lstlisting}
	Dijkstra (entree: noeud x, graphe G) sortie t(x)

	N = {x} //N ensemble de noeuds
	Pour tout (noeud y dans G)
		Si Adj(x,y)= vrai Alors
			D(y) = c(x,y)
			P(y) = x
		Sinon 
			D(y) = +inf
		Fsi
	Fpour

	Repeter
		Trouver z qui n appartient pas N tel que D(z) est minimal
		N = N union {z} //Ajouter z a N

		pour tout (noeud y dans N et Adj(w,y) = vrai)
			D(y) = min( D(y), D(z) + c(z,y) )
			P(y) = z
		fpour
	Jusqu a ce que tous les noeuds soient dans N
	\end{lstlisting}

	\subsubsection{Algorithme à vecteur distance}

	Chaque routeur dispose d'une information locale precisant pour chaque destination connue le meilleur chemin. Le principe est que des informations partielles soient échangées régulièrement entre les routeurs afinde mettre à jour leurs connaissances via des échanges de vecteur. Chaque vecteur associe une destination à son coût estimé.

	\paragraph{Algorithme RIP}\mbox{}\\

	\begin{lstlisting}
	RIP (in:vecteur v, in:routeur R, in/out:table T )

	pour chaque ligne (d,c) de v
		si d n appartient pas a T alors
			ajouter a T (d, c+1, R)
		sinon //deja dans T comme (d,cactuel,lactuel)

			si R = lactuel alors
				mettre a jour T avec (d, c+1, R)
			sinon // R different de lactuel
				si c+1 < cactuel alors
					remplacer dans T avec (d, c+1, R)
				finsi
			finsi
		finsi
	fpour
	\end{lstlisting}

	\emph{Remarques:}
	\begin{description}
		\item[Nœuds terminaux] Les stations de travail ne diffusent pas, seuls les routeurs le font
		\item[A l'initialisation des routeurs] Les tables de routage sont initialisées avec l'ensemble des adresses des réseaux auxquels le routeur est directement connecté, le coût minimum (égal à  1) est alors associé à ces adresses destinations
		\item[Ne tient pas compte du coût réel] Utilise le nombre de sauts (coût de 1)
	\end{description}

	\subsubsection{Routage Hierarchique}

	Internet contient plusieurs milliers de routeur, il est donc impossible d'échanger et de mettre à jour les tables de routage. On a donc recourt à un routage hierarchique: il s'agit zones autonomes reliées par des FAI de plus haut niveau où il y a un routage interne et un routage externe séparés.

	\subsection{Autres algorithmes de routage}

	Tous ces réseaux définissent une structure de trame ainsi que des protocoles d'accès au canal et toutes les fonctionnalités de la couche liaison.\\

	\begin{description}
		\item[Autres classes d'algorithmes de routage] Algorithmes d'états de liens optimisés (notamment utilisé dans le ad-hoc), ou les algorithmes à vecteurs de chemins
		\item[OSPF (Open Shortest Path First)] Un algorithme derivé de Dijkstra
		\item[BGP (Border Gateway Protocol)] Algorithme à vecteur de chemin pour le routage entre les zones autonomes
		\item{Etc.}
	\end{description}

\end{document}